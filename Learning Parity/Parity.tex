% --------------------------------------------------------------
% This is all preamble stuff that you don't have to worry about.
% Head down to where it says ``Start here"
% --------------------------------------------------------------
 
\documentclass[12pt]{article}
 

 
\usepackage[margin=1in]{geometry} 
\usepackage{amsmath,amsthm,amssymb}
\usepackage{graphicx}
\usepackage{tabto}

% \newtheorem{problem}{Problem}
% \newtheorem{solution}{Solution}
 
\newcommand{\N}{\mathbb{N}}
\newcommand{\Z}{\mathbb{Z}}

\newenvironment{solution}[2][Solution]{\begin{trivlist}
\item[\hskip \labelsep {\bfseries #1}\hskip \labelsep {\bfseries #2.}]}{\end{trivlist}}
 
\newenvironment{problem}[2][Problem]{\begin{trivlist}
\item[\hskip \labelsep {\bfseries #1}\hskip \labelsep {\bfseries #2.}]}{\end{trivlist}}
 
\newenvironment{theorem}[2][Theorem]{\begin{trivlist}
\item[\hskip \labelsep {\bfseries #1}\hskip \labelsep {\bfseries #2.}]}{\end{trivlist}}
\newenvironment{claim}[2][Claim]{\begin{trivlist}
		\item[\hskip \labelsep {\bfseries #1}\hskip \labelsep {\bfseries #2.}]}{\end{trivlist}}
\newenvironment{lemma}[2][Lemma]{\begin{trivlist}
\item[\hskip \labelsep {\bfseries #1}\hskip \labelsep {\bfseries #2.}]}{\end{trivlist}}
\newenvironment{exercise}[2][Exercise]{\begin{trivlist}
\item[\hskip \labelsep {\bfseries #1}\hskip \labelsep {\bfseries #2.}]}{\end{trivlist}}
\newenvironment{reflection}[2][Reflection]{\begin{trivlist}
\item[\hskip \labelsep {\bfseries #1}\hskip \labelsep {\bfseries #2.}]}{\end{trivlist}}
\newenvironment{proposition}[2][Proposition]{\begin{trivlist}
\item[\hskip \labelsep {\bfseries #1}\hskip \labelsep {\bfseries #2.}]}{\end{trivlist}}
\newenvironment{corollary}[2][Corollary]{\begin{trivlist}
\item[\hskip \labelsep {\bfseries #1}\hskip \labelsep {\bfseries #2.}]}{\end{trivlist}}
\newenvironment{definition}[2][Definition]{\begin{trivlist}
		\item[\hskip \labelsep {\bfseries #1}\hskip \labelsep {\bfseries #2.}]}{\end{trivlist}}


\newenvironment{example}[2][Example]{\begin{trivlist}
		\item[\hskip \labelsep {\bfseries #1}\hskip \labelsep {\bfseries #2.}]}{\end{trivlist}}
	
\newenvironment{algorithm}[2][Algorithm]{\begin{trivlist}
		\item[\hskip \labelsep {\bfseries #1}\hskip \labelsep {\bfseries #2.}]}{\end{trivlist}}
 
\begin{document}
 
% --------------------------------------------------------------
%                         Start here
% --------------------------------------------------------------
 
%\renewcommand{\qedsymbol}{\filledbox}
 
\title{Learning Parity}
\author{Zhiwei Zhang}
 
\maketitle
\section{Problem Brief}
Given an unknown parity function $$f : \{0, 1\} ^ n \to {0, 1},$$ we want to find a function $g$ that behaves closely to $f$. 

\subsection{Application}
Modeling for finding "revelant" subsets.

\section{Information Theory Perspective}
Given the input bitstring is $"x = x_1x_2 \cdots x_n"$, and the parity function is 
$$f(x) = x_{i_1} + x_{i_2} + \cdots + x_{i_k} \pmod 2 .$$
(Short notice that we actually substitute "$k$" for "$n$" in the original problem brief). \\
\newline
Let $B = \{x_{i_1} , x_{i_2} , \cdots , x_{i_k}\}$\\
\begin{definition}[Notation Definition]{}
Here we specify a notiation: \\
For a input $x$ of length $n$, we use 
$$Parity(x(S)) = \sum\limits_{i \in S}x_i \pmod 2.$$  
Specifically, if $S = B$, then $Parity(x(S)) = Parity(x(B)) = f(x)$
\end{definition}

\noindent We examine that for a random subset of bits $T$ (the indices of the bits), where $|T| = |B| = k$, if we take $s$ uniform random samples of inputs $x^{(1)}, x^{(2)}, \cdots, x^{(s)}$, what's the probability that it behaves exactly the same as the parity function.\\
\newline
We define event $A_s$ for a given subset $S$ as above: $\forall i \in \mathbb{Z} \cap [1, s]$, we have $Parity(x(S)) = f(x)$; and because of the property of binary addition, $P(A_s) = \frac{1}{2^s}$, where each input $x$ has a probability of 1/2 of behaving the same as the parity function.\\
\newline
Since there are at most $n \choose k$ subsets $S != B$, we have:
$$P(\cup_{S}A_{S}) \leq {n \choose k}\frac{1}{2^s}$$
If we want that probability to be less than $1/2$, we will get $s > \log{n \choose k} \approxeq k\log{n}$


\end{document}








