% --------------------------------------------------------------
% This is all preamble stuff that you don't have to worry about.
% Head down to where it says ``Start here"
% --------------------------------------------------------------
 
\documentclass[12pt]{article}
 

 
\usepackage[margin=1in]{geometry} 
\usepackage{amsmath,amsthm,amssymb}
\usepackage{graphicx}
\usepackage{tabto}

% \newtheorem{problem}{Problem}
% \newtheorem{solution}{Solution}
 
\newcommand{\N}{\mathbb{N}}
\newcommand{\Z}{\mathbb{Z}}

\newenvironment{solution}[2][Solution]{\begin{trivlist}
\item[\hskip \labelsep {\bfseries #1}\hskip \labelsep {\bfseries #2.}]}{\end{trivlist}}
 
\newenvironment{problem}[2][Problem]{\begin{trivlist}
\item[\hskip \labelsep {\bfseries #1}\hskip \labelsep {\bfseries #2.}]}{\end{trivlist}}
 
\newenvironment{theorem}[2][Theorem]{\begin{trivlist}
\item[\hskip \labelsep {\bfseries #1}\hskip \labelsep {\bfseries #2.}]}{\end{trivlist}}
\newenvironment{claim}[2][Claim]{\begin{trivlist}
		\item[\hskip \labelsep {\bfseries #1}\hskip \labelsep {\bfseries #2.}]}{\end{trivlist}}
\newenvironment{lemma}[2][Lemma]{\begin{trivlist}
\item[\hskip \labelsep {\bfseries #1}\hskip \labelsep {\bfseries #2.}]}{\end{trivlist}}
\newenvironment{exercise}[2][Exercise]{\begin{trivlist}
\item[\hskip \labelsep {\bfseries #1}\hskip \labelsep {\bfseries #2.}]}{\end{trivlist}}
\newenvironment{reflection}[2][Reflection]{\begin{trivlist}
\item[\hskip \labelsep {\bfseries #1}\hskip \labelsep {\bfseries #2.}]}{\end{trivlist}}
\newenvironment{proposition}[2][Proposition]{\begin{trivlist}
\item[\hskip \labelsep {\bfseries #1}\hskip \labelsep {\bfseries #2.}]}{\end{trivlist}}
\newenvironment{corollary}[2][Corollary]{\begin{trivlist}
\item[\hskip \labelsep {\bfseries #1}\hskip \labelsep {\bfseries #2.}]}{\end{trivlist}}
\newenvironment{definition}[2][Definition]{\begin{trivlist}
		\item[\hskip \labelsep {\bfseries #1}\hskip \labelsep {\bfseries #2.}]}{\end{trivlist}}


\newenvironment{example}[2][Example]{\begin{trivlist}
		\item[\hskip \labelsep {\bfseries #1}\hskip \labelsep {\bfseries #2.}]}{\end{trivlist}}
	
\newenvironment{algorithm}[2][Algorithm]{\begin{trivlist}
		\item[\hskip \labelsep {\bfseries #1}\hskip \labelsep {\bfseries #2.}]}{\end{trivlist}}
 
\begin{document}
 
% --------------------------------------------------------------
%                         Start here
% --------------------------------------------------------------
 
%\renewcommand{\qedsymbol}{\filledbox}
 
\title{Dimension Argument}
\author{UGTCS}
 
\maketitle
\section{Definitions}
$GF(p^k) = $Galois Field of Cardinaly $p^k$.

\section{Random Matrices}
Let $M$ be an $n \times n$ matrix on $GF(2)$. Prove that $Pr[det(M) \not = 0] \geq \frac{1}{4}$

\begin{proof}
	$det(M) \not = 0 \Longleftrightarrow $ Every column is linearly indep.\\
	\newline
	$Pr[\text{each column is LI}] = \prod_{i}Pr[c_1, \cdots, c_i \text{ are LI } | c_1, \cdots, c_{i - 1} \text{ are LI}]$
	$c1, \cdots, c_{i - 1} $ are LI iff sizeof(span($c_1, \cdots, c_{i - 1}$)) = $2^{i - 1}$\\
	$Pr[c_1, \cdots, c_i \text{ are LI } | c_1, \cdots, c_{i - 1} \text{ are LI}] = 1 - \frac{2^{i - 1}}{2 ^ n}$\\
	Use inequality:
	$$a - x \geq 4^{-x} \forall x \in [0, 0.5]$$
\end{proof}


\end{document}








